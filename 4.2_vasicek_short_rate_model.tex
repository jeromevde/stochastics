\documentclass{article}
\usepackage{amsmath}
\usepackage{amssymb}
\usepackage{geometry}
\usepackage[most]{tcolorbox}
\usepackage{hyperref}
\usepackage{xcolor}

\geometry{a4paper, margin=1in}

\begin{document}

\section{Vasicek Short Rate Model}


Interest rates in financial markets come in different forms depending on their time horizon. The \textbf{long rate} $R(t,T)$ represents the continuously compounded zero coupon rate that you can lock in at time $t$ for an investment maturing at time $T$. This is the yield on a bond that pays nothing until maturity.

However, the fundamental building block for interest rate modeling is the \textbf{short rate} $r(t)$, which represents the instantaneous risk-free rate at time $t$. The relationship between these rates is:

\[
r(t) = \lim_{\Delta t \to 0} R(t, t + \Delta t)
\]

In other words, the short rate is the theoretical interest rate for borrowing money for an infinitesimally short period of time.

The long rate $R(t,T)$ can be understood as the market's expectation of the average short rate over the period $[t,T]$, plus a risk premium:

\[
R(t,T) \approx \frac{1}{T-t} \mathbb{E}\left[ \int_t^T r(s) \, ds \right] + \text{Risk Premium}
\]

\subsection{The Vasicek Model}

Vasicek's model directly models the dynamics of the short rate $r(t)$ using the following stochastic differential equation:


\[
\boxed{ dr(t) = \kappa(\theta - r(t)) \, dt + \sigma \, dB(t) }
\]


where $B(t)$ denotes standard Brownian motion.

This model captures two essential features of interest rates:

\begin{itemize}
    \item \textbf{Mean Reversion}: The drift term $\kappa(\theta - r(t))$ pulls the rate toward a long-term mean level $\theta$ at speed $\kappa$.
    \item \textbf{Stochastic Volatility}: The diffusion term $\sigma \, dB(t)$ adds random fluctuations with volatility $\sigma$.
\end{itemize}

The mean reversion property is crucial because interest rates in reality do not follow a random walk. They tend to revert to some equilibrium level due to economic forces and central bank policies.

The model has the following computational advantages:
\begin{itemize}
    \item Analytic formulas for bond prices and bond option prices
    \item Exact simulation of both the short rate and its integral is possible
\end{itemize}

\subsection{Solving the Vasicek SDE}

The Vasicek model is an example of an Ornstein-Uhlenbeck process. We can find an explicit formula for $r(t)$ by solving the SDE.

\begin{center}
    \begin{tcolorbox}[
        colback=lightgray!10, 
        colframe=blue!50!black, 
        boxrule=1.5pt,
        halign=left,
        valign=top,
        rounded corners,
        boxsep=15pt,
        left=15pt, right=15pt, top=15pt, bottom=15pt,
        width=0.9\textwidth
    ]
        \textbf{Proof: Solution of the Vasicek SDE}
        
        \begin{enumerate}
            \item \textbf{Rearrange the SDE:}
            \begin{align*}
                dr(t) &= \kappa(\theta - r(t))dt + \sigma dB(t) \\
                dr(t) + \kappa r(t) dt &= \kappa \theta dt + \sigma dB(t)
            \end{align*}

            \item \textbf{Introduce an Integrating Factor:} We use the integrating factor $e^{\kappa t}$. Let's consider the process $X_t = e^{\kappa t} r(t)$. We can find its differential using Itô's Lemma:
            \[
            d(e^{\kappa t} r(t)) = \kappa e^{\kappa t} r(t) dt + e^{\kappa t} dr(t)
            \]

            \item \textbf{Substitute the SDE:} Now, substitute the expression for $dr(t)$ from step 1:
            \[
            d(e^{\kappa t} r(t)) = \kappa e^{\kappa t} r(t) dt + e^{\kappa t} (\kappa \theta dt - \kappa r(t) dt + \sigma dB(t))
            \]

            \item \textbf{Simplify:} The terms involving $r(t)dt$ cancel out:
            \[
            \begin{aligned}
            d(e^{\kappa t} r(t)) &= \textcolor{gray}{\kappa e^{\kappa t} r(t) dt} + e^{\kappa t} \Big( \kappa \theta dt \textcolor{gray}{- \kappa r(t) dt} + \sigma dB(t) \Big) 
            \end{aligned}
            \]

            \item \textbf{Integrate:} We can now integrate both sides from time 0 to $t$:
            \begin{align*}
                \int_0^t d(e^{\kappa s} r(s)) &= \int_0^t \kappa \theta e^{\kappa s} ds + \int_0^t \sigma e^{\kappa s} dB(s) \\
                e^{\kappa t} r(t) - r(0) &= \theta (e^{\kappa t} - 1) + \sigma \int_0^t e^{\kappa s} dB(s)
            \end{align*}

            \item \textbf{Solve for r(t):} Finally, multiply by $e^{-\kappa t}$ to isolate $r(t)$:
            \[
            \boxed{r(t) = r(0)e^{-\kappa t} + \theta(1 - e^{-\kappa t}) + \sigma \int_0^t e^{-\kappa(t-s)} dB(s)}
            \]
        \end{enumerate}
    \end{tcolorbox}
\end{center}

This is the explicit solution for the Vasicek short rate model. It shows that $r(t)$ is normally distributed, which is a key feature that allows for analytic formulas for bond prices.

\end{document}